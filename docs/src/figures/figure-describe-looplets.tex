\begin{figure}
  \small
  \raggedright

\newcommand{\loopletscale}{0.8}

\begin{minipage}{0.7\linewidth}
  \paragraph{\juliainline{Lookup(body(idx))}}

  An arbitrary sequence of scalars where the element at index i can be computed
  as \juliainline{body(i)}.
\end{minipage}\hspace{8pt}%
\begin{minipage}{0.3\linewidth - 8pt}
\begin{minipage}{\linewidth}
\flushright
\scalebox{\loopletscale}{%
\begin{tikzpicture}[>=latex]
  \matrix (looplet) [matrix of nodes,
              nodes = {indexsty},
              ampersand replacement=\&,
              anchor=north west
              ] at (0,0)
  {
  |[scalarsty]|\juliainline{f(i)} \& ... \& |[scalarsty]|\juliainline{f(j)}\\
  };
  \draw (looplet-1-1.south west) -- ([yshift=-1*\myunit] looplet-1-1.south west);
  \draw[<->] ([yshift=-0.5*\myunit] looplet-1-1.south west) -- ([yshift=-0.5*\myunit] looplet-1-3.south east) node[midway, fill=white]{\juliainline{i:j}};
  \draw (looplet-1-3.south east) -- ([yshift=-1*\myunit] looplet-1-3.south east);

\end{tikzpicture}%
}
\end{minipage}
\end{minipage}

\begin{minipage}{0.7\linewidth}
  \paragraph{\juliainline{Run(body)}}

  A sequence of the same repeated scalar \juliainline{body}.
\end{minipage}\hspace{8pt}%
\begin{minipage}{0.3\linewidth - 8pt}
\begin{minipage}{\linewidth}
\flushright
  \scalebox{\loopletscale}{%
  \begin{tikzpicture}[>=latex]
    \matrix (looplet) [matrix of nodes,
                nodes = {indexsty},
                ampersand replacement=\&,
                anchor=north west
                ] at (0,0)
    {
  |[scalarsty]|\juliainline{x} \& ... \& |[scalarsty]|\juliainline{x}\\
    };
    \draw (looplet-1-1.south west) -- ([yshift=-1*\myunit] looplet-1-1.south west);
    \draw[<->] ([yshift=-0.5*\myunit] looplet-1-1.south west) -- ([yshift=-0.5*\myunit] looplet-1-3.south east) node[midway, fill=white]{\juliainline{i:j}};
    \draw (looplet-1-3.south east) -- ([yshift=-1*\myunit] looplet-1-3.south east);
  
  \end{tikzpicture}%
  }
\end{minipage}
\end{minipage}

\begin{minipage}{0.7\linewidth}
  \paragraph{\juliainline{Spike(body, tail)}}

  A sequence of the same repeated scalar \juliainline{body}, followed by some
  scalar \juliainline{tail} at \juliainline{stop}.
\end{minipage}\hspace{8pt}%
\begin{minipage}{0.3\linewidth - 8pt}
\begin{minipage}{\linewidth}
\flushright
  \scalebox{\loopletscale}{%
  \begin{tikzpicture}[>=latex]
    \matrix (looplet) [matrix of nodes,
                nodes = {indexsty},
                ampersand replacement=\&,
                anchor=north west
                ] at (0,0)
    {
  |[scalarsty]|\juliainline{z} \& ... \& |[scalarsty]|\juliainline{z} \& |[scalarsty]|\juliainline{x}\\
    };
    \draw (looplet-1-1.south west) -- ([yshift=-1*\myunit] looplet-1-1.south west);
    \draw[<->] ([yshift=-0.5*\myunit] looplet-1-1.south west) -- ([yshift=-0.5*\myunit] looplet-1-3.south east) node[midway, fill=white]{\juliainline{i:j-1}};
    \draw (looplet-1-3.south east) -- ([yshift=-1*\myunit] looplet-1-3.south east);
    \draw[<->] ([yshift=-0.5*\myunit] looplet-1-4.south west) -- ([yshift=-0.5*\myunit] looplet-1-4.south east) node[midway, fill=white]{\juliainline{j}};
    \draw (looplet-1-4.south east) -- ([yshift=-1*\myunit] looplet-1-4.south east);
  
  \end{tikzpicture}%
  }
\end{minipage}
\end{minipage}

  \paragraph{\juliainline{Pipeline(Phase(stride, body(ext)), ...)}}

  The concatenation of a few different child looplets in a sequence. Each phase
  declares its corresponding extent, ending at \juliainline{stride}, and the
  child \juliainline{body(ext)} for some target subextent.

\begin{minipage}{\linewidth}
  \flushright
\scalebox{\loopletscale}{%
\begin{tikzpicture}[>=latex]
  \matrix (looplet) [matrix of nodes,
              nodes = {indexsty},
              ampersand replacement=\&,
              anchor=north west
              ] at (0,0)
  {
|[scalarsty, fill=blue!50,minimum width=6*\myunit]|\juliainline{A} \& |[scalarsty, fill=red!50,minimum width=3*\myunit]|\juliainline{B}\\
  };

  \draw ([xshift=\myunit]looplet-1-1.north west) -- ([yshift=1*\myunit, xshift=\myunit] looplet-1-1.north west);
  \draw[<->] ([xshift=1*\myunit, yshift=0.5*\myunit] looplet-1-1.north west) -- ([xshift=-1*\myunit, yshift=0.5*\myunit] looplet-1-1.north east) node[midway, fill=white]{\juliainline{sub_i:sub_j}};
  \draw ([xshift=-1*\myunit]looplet-1-1.north east) -- ([xshift=-1*\myunit, yshift=1*\myunit] looplet-1-1.north east);

  \draw (looplet-1-1.south west) -- ([yshift=-1*\myunit] looplet-1-1.south west);
  \draw[<->] ([yshift=-0.5*\myunit] looplet-1-1.south west) -- ([yshift=-0.5*\myunit] looplet-1-1.south east) node[midway, fill=white]{\juliainline{i:k}};
  \draw (looplet-1-1.south east) -- ([yshift=-1*\myunit] looplet-1-1.south east);
  \draw[<->] ([yshift=-0.5*\myunit] looplet-1-2.south west) -- ([yshift=-0.5*\myunit] looplet-1-2.south east) node[midway, fill=white]{\juliainline{k + 1:j}};
  \draw (looplet-1-2.south east) -- ([yshift=-1*\myunit] looplet-1-2.south east);

\end{tikzpicture}%
}
\end{minipage}

  \paragraph{\juliainline{Stepper(seek(idx), stride, body(ext), next)}}

  The repeated application of the same child looplet \juliainline{body(ext)},
  each child ending at \juliainline{stride}. Steppers are evaluated iteratively,
  so \juliainline{next} advances state to the next looplet and
  \juliainline{seek(i)} initializes variables for a starting index
  \juliainline{i}.
\begin{minipage}{\linewidth}
\flushright
\scalebox{\loopletscale}{%
\begin{tikzpicture}[>=latex]
  \matrix (looplet) [matrix of nodes,
              nodes = {indexsty},
              ampersand replacement=\&,
              anchor=north west
              ] at (0,0)
  {
|[scalarsty, minimum width=6*\myunit]|\juliainline{A_1} \& |[scalarsty, minimum width=2*\myunit]|\juliainline{A_2} \&... \& |[scalarsty, minimum width=3*\myunit]|\juliainline{A_n}\\
  };

  \draw ([xshift=\myunit]looplet-1-1.north west) -- ([yshift=1*\myunit, xshift=\myunit] looplet-1-1.north west);
  \draw[<->] ([xshift=1*\myunit, yshift=0.5*\myunit] looplet-1-1.north west) -- ([xshift=-1*\myunit, yshift=0.5*\myunit] looplet-1-1.north east) node[midway, fill=white]{\juliainline{sub_i:sub_j}};
  \draw ([xshift=-1*\myunit]looplet-1-1.north east) -- ([xshift=-1*\myunit, yshift=1*\myunit] looplet-1-1.north east);

  \draw (looplet-1-1.south west) -- ([yshift=-1*\myunit] looplet-1-1.south west);
  \draw[<->] ([yshift=-0.5*\myunit] looplet-1-1.south west) -- ([yshift=-0.5*\myunit] looplet-1-1.south east) node[midway, fill=white]{\juliainline{i:k}};
  \draw (looplet-1-1.south east) -- ([yshift=-1*\myunit] looplet-1-1.south east);
  \draw[<->] ([yshift=-0.5*\myunit] looplet-1-1.south east) -- ([yshift=-0.5*\myunit] looplet-1-4.south east) node[midway, fill=white]{\juliainline{k + 1:j}};
  \draw (looplet-1-4.south east) -- ([yshift=-1*\myunit] looplet-1-4.south east);

\end{tikzpicture}%
}
\end{minipage}

  \paragraph{\juliainline{Jumper(seek(idx), stride, body(ext), next)}}

  Like a stepper, but \juliainline{ext} may be wider than the extent declared by
  \juliainline{stride}, enabling accelerated iteration (e.g. galloping).

\begin{minipage}{\linewidth}
\flushright
\scalebox{\loopletscale}{%
\begin{tikzpicture}[>=latex]
  \matrix (looplet) [matrix of nodes,
              nodes = {indexsty},
              ampersand replacement=\&,
              anchor=north west
              ] at (0,0)
  {
|[scalarsty, minimum width=6*\myunit]|\juliainline{A_1} \& |[scalarsty, minimum width=2*\myunit]|\juliainline{A_2} \&... \& |[scalarsty, minimum width=3*\myunit]|\juliainline{A_n}\\
  };

  \draw ([xshift=\myunit]looplet-1-1.north west) -- ([yshift=1*\myunit, xshift=\myunit] looplet-1-1.north west);
  \draw[<->] ([xshift=1*\myunit, yshift=0.5*\myunit] looplet-1-1.north west) -- ([xshift=-1*\myunit, yshift=0.5*\myunit] looplet-1-2.north east) node[midway, fill=white]{\juliainline{sub_i:sub_j}};
  \draw ([xshift=-1*\myunit]looplet-1-2.north east) -- ([xshift=-1*\myunit, yshift=1*\myunit] looplet-1-2.north east);

  \draw (looplet-1-1.south west) -- ([yshift=-1*\myunit] looplet-1-1.south west);
  \draw[<->] ([yshift=-0.5*\myunit] looplet-1-1.south west) -- ([yshift=-0.5*\myunit] looplet-1-1.south east) node[midway, fill=white]{\juliainline{i:k}};
  \draw (looplet-1-1.south east) -- ([yshift=-1*\myunit] looplet-1-1.south east);
  \draw[<->] ([yshift=-0.5*\myunit] looplet-1-1.south east) -- ([yshift=-0.5*\myunit] looplet-1-4.south east) node[midway, fill=white]{\juliainline{k + 1:j}};
  \draw (looplet-1-4.south east) -- ([yshift=-1*\myunit] looplet-1-4.south east);

\end{tikzpicture}%
}
\end{minipage}

\vspace{10pt}

\begin{minipage}{0.5\linewidth}
  \paragraph{\juliainline{Shift(delta, body)}}

  A wrapper that shifts all declared extents of \juliainline{body} by
  \juliainline{delta}.  Shifting is necessary because extents are
  absolute, rather than relative.
\end{minipage}\hspace{8pt}%
\begin{minipage}{0.5\linewidth - 8pt}
\begin{minipage}{\linewidth}
\flushright
\scalebox{\loopletscale}{%
\begin{tikzpicture}[>=latex]
  \matrix (looplet) [matrix of nodes,
              nodes = {indexsty},
              ampersand replacement=\&,
              anchor=north west
              ] at (0,0)
  {
  |[scalarsty, minimum width = 4*\myunit]|\juliainline{A}\\
  };
  \draw (looplet-1-1.south west) -- ([yshift=-2*\myunit] looplet-1-1.south west);
  \draw[dotted] (looplet-1-1.south west) -- ([xshift=3*\myunit, yshift=-1*\myunit] looplet-1-1.south west);
  \draw[dotted] (looplet-1-1.south east) -- ([xshift=3*\myunit, yshift=-1*\myunit] looplet-1-1.south east);
  \draw ([xshift=3*\myunit, yshift=-1*\myunit]looplet-1-1.south west) -- ([xshift=3*\myunit, yshift=-2*\myunit] looplet-1-1.south west);
  \draw[<->] ([xshift=3*\myunit, yshift=-1.5*\myunit] looplet-1-1.south west) -- ([xshift=3*\myunit, yshift=-1.5*\myunit] looplet-1-1.south east) node[midway, fill=white]{\juliainline{i:j}};
  \draw[<->] ([yshift=-1.5*\myunit] looplet-1-1.south west) -- ([xshift=3*\myunit, yshift=-1.5*\myunit] looplet-1-1.south west) node[midway, fill=white]{\juliainline{delta}};
  \draw ([xshift=3*\myunit, yshift=-1*\myunit]looplet-1-1.south east) -- ([xshift=3*\myunit, yshift=-2*\myunit] looplet-1-1.south east);

\end{tikzpicture}%
}
\end{minipage}
\end{minipage}

\begin{minipage}{0.55\linewidth}
  \paragraph{\juliainline{Switch(Case(cond, body), ...)}}

  Represents the first child looplet \juliainline{body} for which
  \juliainline{cond} evaluates to true at runtime.

\end{minipage}\hspace{8pt}%
\begin{minipage}{0.45\linewidth - 8pt}
\begin{minipage}{\linewidth}
\flushright
\scalebox{\loopletscale}{%
\begin{tikzpicture}[>=latex]
  \matrix (looplet) [matrix of nodes,
              nodes = {indexsty},
              ampersand replacement=\&,
              anchor=north west
              ] at (0,0)
  {
  |[minimum width = 2.0*\myunit]|\juliainline{if cond} \& |[scalarsty, fill=red!50, minimum width = 4*\myunit]|\juliainline{A}\\
  |[minimum width = 2.0*\myunit]|\juliainline{else} \& |[scalarsty, fill=blue!50, minimum width = 4*\myunit]|\juliainline{B}\\
  };
  \draw[lpltsty] (looplet-1-1.north west) -- (looplet-2-1.south west);
  \draw (looplet-2-2.south west) -- ([yshift=-1*\myunit] looplet-2-2.south west);
  \draw[<->] ([yshift=-0.5*\myunit] looplet-2-2.south west) -- ([yshift=-0.5*\myunit] looplet-2-2.south east) node[midway, fill=white]{\juliainline{i:j}};
  \draw (looplet-2-2.south east) -- ([yshift=-1*\myunit] looplet-2-2.south east);

\end{tikzpicture}%
}
\end{minipage}
\end{minipage}
  \caption{The looplets considered in this paper, described and displayed with a target extent of \juliainline{i:j}.}\label{fig:looplets}
\end{figure}