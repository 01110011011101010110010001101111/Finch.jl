\begin{figure}
\begin{minipage}[t]{0.5\linewidth - 2pt}

\begin{subfigure}[t]{\linewidth}
\scalebox{\formatscale}{%
\begin{tikzpicture}[>=latex]
  \matrix (vals) [matrix of math nodes,
              nodes = {whclsty},
              ampersand replacement=\&,
              anchor=north west] at (0,0)
  {
  ... \& |[nzsty]|1.9 \& |[nzsty]|3.0 \& |[nzsty]|2.7 \& |[nzsty]|5.5 \& ... \\
  };
  \node [draw, whclsty, left=of vals] {\juliainline{.val}};

  \matrix (idx) [matrix of math nodes,
              nodes = {whclsty},
              ampersand replacement=\&,
              anchor=north west] at (0,1.5 * \myunit)
  {
  ... \& 2 \& 4 \& 5 \& 9 \& ... \\
  };
  \node [draw, whclsty, left=of idx] {\juliainline{.idx}};

  \matrix (pos) [matrix of math nodes,
              nodes = {whclsty},
              ampersand replacement=\&,
              anchor=north west] at (0,3 * \myunit)
  {
  ... \& |[nzsty]|18 \& |[nzsty]| 23 \& ... \\
  };
  \node [draw, whclsty, left=of pos] {\juliainline{.pos}};

\end{tikzpicture}%
}
\resizebox{\linewidth}{!}{%
\begin{tikzpicture}[>=latex]
\matrix (A) [matrix of math nodes,
  nodes = {whclsty},
  left delimiter  = (,
  right delimiter = ),
  ampersand replacement=\&] at (0,0)
{
  \zecl \& |[nzsty]|1.9 \& \zecl \& |[nzsty]|3.0 \& |[nzsty]|2.7 \& \zecl \& \zecl \& \zecl \& |[nzsty]|5.5 \& \zecl \& \zecl \\
};

\draw[lpltsty] ([yshift=-0*\myunit]A-1-1.south west) -- ([yshift=-0*\myunit]A-1-1.south east) node[namesty]{Run};
%\draw[lpltsty] ([yshift=-0*\myunit]A-1-2.south west) -- ([yshift=-0*\myunit]A-1-2.south east) node[namesty]{Lookup};
\draw[lpltsty] ([yshift=-0*\myunit]A-1-3.south west) -- ([yshift=-0*\myunit]A-1-3.south east) node[namesty]{Run};
%\draw[lpltsty] ([yshift=-0*\myunit]A-1-4.south west) -- ([yshift=-0*\myunit]A-1-4.south east) node[namesty]{Lookup};
%\draw[lpltsty] ([yshift=-0*\myunit]A-1-5.south west) -- ([yshift=-0*\myunit]A-1-5.south east) node[namesty]{Lookup};
\draw[lpltsty] ([yshift=-0*\myunit]A-1-6.south west) -- ([yshift=-0*\myunit]A-1-8.south east) node[namesty]{Run};
%\draw[lpltsty] ([yshift=-0*\myunit]A-1-9.south west) -- ([yshift=-0*\myunit]A-1-9.south east) node[namesty]{Lookup};
\draw[lpltsty] ([yshift=-0*\myunit]A-1-10.south west) -- ([yshift=-0*\myunit]A-1-11.south east) node[namesty]{Run};

\draw[lpltsty] ([yshift=-1*\myunit]A-1-1.south west) -- ([yshift=-1*\myunit]A-1-2.south east) node[namesty]{Spike};
\draw[lpltsty] ([yshift=-1*\myunit]A-1-3.south west) -- ([yshift=-1*\myunit]A-1-4.south east) node[namesty]{Spike};
\draw[lpltsty] ([yshift=-1*\myunit]A-1-5.south west) -- ([yshift=-1*\myunit]A-1-5.south east) node[namesty]{Spike};
\draw[lpltsty] ([yshift=-1*\myunit]A-1-6.south west) -- ([yshift=-1*\myunit]A-1-9.south east) node[namesty]{Spike};

\draw[lpltsty] ([yshift=-2*\myunit]A-1-1.south west) -- ([yshift=-2*\myunit]A-1-9.south east) node[namesty]{Stepper};
\draw[lpltsty] ([yshift=-3*\myunit]A-1-1.south west) -- ([yshift=-3*\myunit]A-1-9.south east) node[namesty]{Jumper};

\draw[lpltsty] ([yshift=-4*\myunit]A-1-1.south west) -- ([yshift=-4*\myunit]A-1-11.south east) node[namesty]{Pipeline};

\end{tikzpicture}%
}
\begin{juliacode}
Pipeline(
  Phase(
    stride = idx[pos[i+1]-1],
    body = begin
      p = pos[i]
      Jumper(
        seek(j) = (p = search(idx, j)),
        stride = idx[p],
        body(j) = Switch(
          Case(
            cond = idx[p] == j,
            body = Spike(
              body = Run(0)
              tail = val[p])),
          Case(
            body = Stepper(
              seek(j) =
                (p = search(idx, j)),
              stride = idx[p],
              body = Spike(
                body = Run(0)
                tail = val[p]),
              next = p += 1))),
        next = p += 1)
    end),
  Phase(
    body = Run(0)))
\end{juliacode}
\caption{A galloping (leader) protocol for a list format. Compare to the walking
(follower) protocol of Figure \ref{fig:structure:uniform-list-walk}.}\label{fig:proto:gallop}
\end{subfigure}

\end{minipage}\hspace{4pt}%
\begin{minipage}[t]{0.5\linewidth - 2pt}

\begin{subfigure}[t]{\linewidth}
\resizebox{\linewidth}{!}{\scalebox{\formatscale}{%
\begin{tikzpicture}[>=latex]
  \matrix (vals) [matrix of math nodes,
              nodes = {whclsty},
              ampersand replacement=\&,
              anchor=north west] at (0,0)
  {
  ... \& |[nzsty]|0.0 \& |[nzsty]|1.9 \& |[nzsty]|0.0 \& |[nzsty]|3.0 \& |[nzsty]|2.7 \& |[nzsty]|0.0 \& |[nzsty]|0.0 \& |[nzsty]|0.0 \& |[nzsty]|5.5 \& |[nzsty]|0.0 \& |[nzsty]|0.0 \& ...\\
  };

  \node [draw, whclsty, anchor=north west] at (0, 1.5*\myunit) {\juliainline{.val}};
\end{tikzpicture}%
}}
\resizebox{\linewidth}{!}{%
\begin{tikzpicture}[>=latex]
\matrix (A) [matrix of math nodes,
  nodes = {nzsty},
  left delimiter  = (,
  right delimiter = ),
  ampersand replacement=\&] at (0,0)
{
  0 \& 1.9 \& 0 \& 3.0 \& 2.7 \& 0 \& 0 \& 0 \& 5.5 \& 0 \& 0 \\
};

\draw[lpltsty] ([yshift=-0*\myunit]A-1-1.south west) -- ([yshift=-0*\myunit]A-1-11.south east) node[namesty]{Lookup};

\end{tikzpicture}%
}
\begin{juliacode}
Pipeline(
  Lookup(
    body(j) = val[i*n+j]))
\end{juliacode}
\caption{A locate protocol for a dense format. All entries are treated as if they might be nonzero.}\label{fig:proto:dense}
\end{subfigure}

\begin{subfigure}[t]{\linewidth}
\resizebox{\linewidth}{!}{\scalebox{\formatscale}{%
\begin{tikzpicture}[>=latex]
  \matrix (vals) [matrix of math nodes,
              nodes = {whclsty},
              ampersand replacement=\&,
              anchor=north west] at (0,0)
  {
  ... \& ? \& |[nzsty]|1.9 \& ? \& |[nzsty]|3.0 \& |[nzsty]|2.7 \& ? \& ? \& ? \& |[nzsty]|5.5 \& ? \& ? \& ...\\
  };
  \node [draw, whclsty, anchor=north west] at (0, 1.5*\myunit) {\juliainline{.val}};

  \matrix (tbl) [matrix of math nodes,
              nodes = {whclsty},
              ampersand replacement=\&,
              anchor=north west] at (0,-3*\myunit)
  {
  ... \& |[nzsty]|0 \& |[nzsty]|1 \& |[nzsty]|0 \& |[nzsty]|1 \& |[nzsty]|1 \& |[nzsty]|0 \& |[nzsty]|0 \& |[nzsty]|0 \& |[nzsty]|1 \& |[nzsty]|0 \& |[nzsty]|0 \& ...\\
  };

  \node [draw, whclsty, anchor=north west] at (0, -1.5*\myunit) {\juliainline{.tbl}};
\end{tikzpicture}%
}}
\resizebox{\linewidth}{!}{%
\begin{tikzpicture}[>=latex]
\matrix (A) [matrix of math nodes,
  nodes = {whclsty},
  left delimiter  = (,
  right delimiter = ),
  ampersand replacement=\&] at (0,0)
{
  \zecl \& |[nzsty]|1.9 \& \zecl \& |[nzsty]|3.0 \& |[nzsty]|2.7 \& \zecl \& \zecl \& \zecl \& |[nzsty]|5.5 \& \zecl \& \zecl \\
};

\draw[lpltsty] ([yshift=-0*\myunit]A-1-1.south west) -- ([yshift=-0*\myunit]A-1-1.south east) node[namesty]{\rotatebox{45}{Switch}};
\draw[lpltsty] ([yshift=-0*\myunit]A-1-2.south west) -- ([yshift=-0*\myunit]A-1-2.south east) node[namesty]{\rotatebox{45}{Switch}};
\draw[lpltsty] ([yshift=-0*\myunit]A-1-3.south west) -- ([yshift=-0*\myunit]A-1-3.south east) node[namesty]{\rotatebox{45}{Switch}};
\draw[lpltsty] ([yshift=-0*\myunit]A-1-4.south west) -- ([yshift=-0*\myunit]A-1-4.south east) node[namesty]{\rotatebox{45}{Switch}};
\draw[lpltsty] ([yshift=-0*\myunit]A-1-5.south west) -- ([yshift=-0*\myunit]A-1-5.south east) node[namesty]{\rotatebox{45}{Switch}};
\draw[lpltsty] ([yshift=-0*\myunit]A-1-6.south west) -- ([yshift=-0*\myunit]A-1-6.south east) node[namesty]{\rotatebox{45}{Switch}};
\draw[lpltsty] ([yshift=-0*\myunit]A-1-7.south west) -- ([yshift=-0*\myunit]A-1-7.south east) node[namesty]{\rotatebox{45}{Switch}};
\draw[lpltsty] ([yshift=-0*\myunit]A-1-8.south west) -- ([yshift=-0*\myunit]A-1-8.south east) node[namesty]{\rotatebox{45}{Switch}};
\draw[lpltsty] ([yshift=-0*\myunit]A-1-9.south west) -- ([yshift=-0*\myunit]A-1-9.south east) node[namesty]{\rotatebox{45}{Switch}};
\draw[lpltsty] ([yshift=-0*\myunit]A-1-10.south west) -- ([yshift=-0*\myunit]A-1-10.south east) node[namesty]{\rotatebox{45}{Switch}};
\draw[lpltsty] ([yshift=-0*\myunit]A-1-11.south west) -- ([yshift=-0*\myunit]A-1-11.south east) node[namesty]{\rotatebox{45}{Switch}};
\draw[lpltsty] ([yshift=-2*\myunit]A-1-1.south west) -- ([yshift=-2*\myunit]A-1-11.south east) node[namesty]{Lookup};

\end{tikzpicture}%
}
\begin{juliacode}
Pipeline(
  Lookup(
    body(j) = Switch(
      Case(
        cond = tbl[i*n+j],
        body = val[i*n+j]),
      Case(0))))
\end{juliacode}
\caption{A locate protocol for a bitmap format. The switch branches on whether each value is statically zero.}\label{fig:proto:bitmap}
\end{subfigure}
\end{minipage}

\caption{A protocol language also allows us to iterate over the same structure
(or even the same format) in different ways, enabling new optimization
opportunities.}
\end{figure}